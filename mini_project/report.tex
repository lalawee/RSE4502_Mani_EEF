\documentclass[12pt]{article}

\usepackage{amsmath}
\usepackage{amssymb}
\usepackage{graphicx}
\usepackage{geometry}
\usepackage{placeins}
\geometry{margin=1in}

\title{RSE4502 Mini Project 1\\
Jacobian-Based Welding Trajectory Tracking}
\date{\today}

\begin{document}

\maketitle



\section{Project Summary}
The objective of this project is to simulate a robotic manipulator tracing 
a circular welding seam around a cylindrical workpiece of radius
\[
r = 0.5~\text{m},
\]
whose seam lies at a fixed height of
\[
h_t = 1.0~\text{m}.
\]
The end-effector is required to follow this circular trajectory while 
maintaining a constant tool orientation relative to the workpiece.

Rather than implementing a controller for a single fixed robot design, 
the software is structured as a configuration-agnostic framework. The 
system dynamically constructs a robotic manipulator from user-defined 
Modified Denavit--Hartenberg (Craig) parameters and specified joint types 
(Revolute, Prismatic, or Fixed). This allows different robot architectures 
to be evaluated under identical task conditions by changing a single 
configuration string.

The framework is composed of four major subsystems:

\begin{itemize}
    \item \textbf{Dynamic Robot Construction:}  
    A \texttt{rigidBodyTree} model is built automatically from Modified 
    DH parameters and joint metadata. Joint limits and velocity limits 
    are enforced at all stages of simulation.

    \item \textbf{Initial Configuration Search:}
    A valid starting joint configuration is found using MATLAB's 
    \texttt{inverseKinematics} solver, initialised from a manually 
    specified posture hint that guides the solver toward a geometrically 
    sensible arm configuration. The solution is verified by checking 
    position error and manipulability at the returned configuration.

    \item \textbf{Jacobian-Based Inverse Kinematics:}  
    A differential inverse kinematics loop is implemented using the 
    geometric Jacobian, computed from first principles via direct column 
    construction. Joint velocities are resolved from desired end-effector 
    velocities using a Damped Least Squares (DLS) formulation to provide 
    robustness near singular configurations. A proportional position 
    feedback term is included to correct accumulated integration drift 
    and maintain sub-millimetre steady-state tracking error.

    \item \textbf{Visualisation and Replay:}  
    The resulting motion, tracking error, singularity metrics, and robot 
    behaviour are visualised and animated for evaluation.
\end{itemize}

Additionally, the framework explicitly accounts for joint limits and the 
robot base offset relative to the centre of the cylindrical workpiece. 
This enables systematic evaluation of different robot configurations 
without requiring structural changes to downstream modules.

\section{Dynamic Rigid-Body Construction via Modified DH (Craig)}

Robot models are constructed programmatically from a configuration string
without hard-coding link objects. Each configuration specifies a Modified
DH (Craig) table with explicit joint types per row:
\[
\text{DH row } i:\ \big[\alpha_{i-1}~(\deg),\ a_{i-1}~(\text{m}),\
d_i~(\text{m}),\ \theta_{0,i}~(\deg),\ \text{type} \in \{R, P, F\}\big].
\]

\subsection{Robot Configuration: Modified JetArm (PRRRRR)}

The original JetArm is a five-revolute (RRRRR) manipulator. A prismatic
joint is prepended at the base for vertical lift, satisfying the
assignment requirement and extending vertical reach to cover
$h_t = 1.0~\text{m}$. All link lengths are scaled by $6\times$ from the
physical platform:
\[
l_0 = 0.619~\text{m},\quad l_1 = l_2 = 0.777~\text{m},\quad
l_3 = 0.357~\text{m},\quad l_4 = 0.153~\text{m}.
\]
The base is offset $b_x = 1.2~\text{m}$ from the cylinder centre.

\begin{table}[h]
\centering
\small
\begin{tabular}{cccccl}
\hline
$i$ & $\alpha_{i-1}$ & $a_{i-1}$ (m) & $d_i$ (m) &
$\theta_{\text{off}}$ (deg) & Type \\
\hline
1 & $0$   & $0$   & $0$   & $0$   & P (prismatic lift) \\
2 & $0$   & $0$   & $l_0$ & $0$   & R (base yaw) \\
3 & $-90$ & $0$   & $0$   & $-90$ & R (shoulder) \\
4 & $0$   & $l_1$ & $0$   & $0$   & R (elbow) \\
5 & $0$   & $l_2$ & $0$   & $-90$ & R (wrist pitch) \\
6 & $-90$ & $0$   & $l_3$ & $0$   & R (wrist roll) \\
7 & $0$   & $0$   & $l_4$ & $180$ & F (fixed tool) \\
\hline
\end{tabular}
\caption{DH parameters for the modified JetArm (PRRRRR).}
\label{tab:dh_jetarm}
\end{table}

Joint limits and velocity limits are given in Table~\ref{tab:limits_jetarm}.
The yaw and shoulder joints permit $\pm 2\pi$ to allow continuous
circumferential sweeping. Velocity limits are set high relative to task
demands --- peak observed velocities were under $0.4~\text{rad/s}$, making
them non-binding.

\begin{table}[h]
\centering
\small
\begin{tabular}{clccc}
\hline
Joint & Description & $q_{\min}$ & $q_{\max}$ & $\dot{q}_{\max}$ \\
\hline
1 & Prismatic lift & $-0.5~\text{m}$    & $1.5~\text{m}$    & $10~\text{m/s}$ \\
2 & Base yaw       & $-2\pi~\text{rad}$ & $2\pi~\text{rad}$ & $10~\text{rad/s}$ \\
3 & Shoulder       & $-2\pi~\text{rad}$ & $2\pi~\text{rad}$ & $10~\text{rad/s}$ \\
4 & Elbow          & $-\pi~\text{rad}$  & $\pi~\text{rad}$  & $10~\text{rad/s}$ \\
5 & Wrist pitch    & $-\pi~\text{rad}$  & $\pi~\text{rad}$  & $10~\text{rad/s}$ \\
6 & Wrist roll     & $-\pi~\text{rad}$  & $\pi~\text{rad}$  & $10~\text{rad/s}$ \\
\hline
\end{tabular}
\caption{Joint and velocity limits for the modified JetArm.}
\label{tab:limits_jetarm}
\end{table}

\subsection{Robot Assembly (\texttt{buildConfig} and \texttt{buildRobotFromDH})}

\texttt{buildConfig} maps the configuration string to a DH cell array and
extracts the numeric DH matrix, joint type list, joint limits
$\mathbf{jLimits}$, velocity limits $\mathbf{vLimits}$, and posture hint
$\mathbf{q}_{\text{hint}}$. For revolute joints $q_k$ adds to $\theta$;
for prismatic joints $q_k$ adds to $d$.

\texttt{buildRobotFromDH} iterates through each DH row, creates a
\texttt{rigidBody} and \texttt{rigidBodyJoint} of the appropriate type,
computes the fixed transform via \texttt{spatialTransform}, and appends
the body to the kinematic chain. Joint limits are applied after
construction, with home position set to $q_{k,\text{home}} =
(q_{k,\min} + q_{k,\max})/2$.

% =====================================================================
\subsection{Joint Limit Application and Home Positioning}
% =====================================================================

After the tree is built, joint limits are applied to each active joint in
sequence (fixed joints are skipped). For joint $k$:
\[
q_{k,\min} \le q_k \le q_{k,\max}.
\]
To ensure the \texttt{rigidBodyTree} has a valid default pose, the home
position is set to the midpoint of the allowable range:
\[
q_{k,\text{home}} = \frac{q_{k,\min} + q_{k,\max}}{2}.
\]

Joint limits and velocity limits for the modified JetArm are:

\begin{table}[h]
\centering
\begin{tabular}{clccc}
\hline
Joint & Description & $q_{\min}$ & $q_{\max}$ & $\dot{q}_{\max}$ \\
\hline
1 & Prismatic lift & $-0.5~\text{m}$    & $1.5~\text{m}$    & $10~\text{m/s}$ \\
2 & Base yaw       & $-2\pi~\text{rad}$ & $2\pi~\text{rad}$ & $10~\text{rad/s}$ \\
3 & Shoulder pitch & $-2\pi~\text{rad}$ & $2\pi~\text{rad}$ & $10~\text{rad/s}$ \\
4 & Elbow pitch    & $-\pi~\text{rad}$  & $\pi~\text{rad}$  & $10~\text{rad/s}$ \\
5 & Wrist pitch    & $-\pi~\text{rad}$  & $\pi~\text{rad}$  & $10~\text{rad/s}$ \\
6 & Wrist roll     & $-\pi~\text{rad}$  & $\pi~\text{rad}$  & $10~\text{rad/s}$ \\
\hline
\end{tabular}
\caption{Joint limits and velocity limits for the modified JetArm (PRRRRR).}
\label{tab:limits_jetarm}
\end{table}

Joint limits were selected based on geometric reachability rather than
hardware constraints, since the robot is simulated. The prismatic lift is
bounded to $[-0.5,\, 1.5]~\text{m}$, centring the seam height
$h_t = 1.0~\text{m}$ within the stroke with $\pm 0.5~\text{m}$ margin.
The yaw and shoulder joints are permitted full multi-turn rotation
($\pm 2\pi$) since the task requires continuous circumferential sweeping
without encountering a joint stop. Velocity limits are set high relative
to task demands — the unconstrained simulation showed peak velocities well
below these values, and tighter limits had no measurable effect on tracking
performance.

The final outputs of this module are:
\[
(\texttt{robot},\ \mathbf{DH},\ \texttt{jointTypes},\ \mathbf{jLimits},\
\mathbf{vLimits},\ \mathbf{q}_{\text{hint}}).
\]

\section{Pre-Execution Setup}

Prior to executing the Jacobian inverse kinematics loop, two preparatory 
steps are performed: (1) the welding path is defined in the robot base frame 
as the reference trajectory for the IK loop, and (2) a valid initial joint 
configuration is found by a manipulability-aware forward kinematics search. 
Together these steps ensure that the main simulation loop begins with a 
well-defined reference signal and a well-conditioned starting state.

% =========================================================
\section{Welding Path Definition}
% =========================================================

The welding path is defined as a constant-angular-velocity circular seam 
of radius $r = 0.5~\text{m}$ at height $h_t = 1.0~\text{m}$, centred at 
the world origin. Since the Jacobian and all kinematic computations are 
expressed in the robot base frame, the path must be transformed accordingly 
before being used as a reference for the IK loop.

\subsection{Coordinate Frame Transformation}

The robot base is offset from the world origin by $b_x$ along the $X$-axis. 
The seam position in the world frame is

\[
\mathbf{p}_{\text{world}}(t) =
\begin{bmatrix}
r\cos(\omega t) \\
r\sin(\omega t) \\
h_t
\end{bmatrix},
\]

and after subtracting the base offset, the path expressed in the robot base 
frame is

\[
\mathbf{p}_{\text{robot}}(t) =
\begin{bmatrix}
r\cos(\omega t) - b_x \\
r\sin(\omega t) \\
h_t
\end{bmatrix}.
\]

Since $b_x$ is constant, it vanishes upon differentiation and the desired 
linear velocity in the robot base frame is

\[
\dot{\mathbf{p}}_{\text{robot}}(t) =
\begin{bmatrix}
-r\omega\sin(\omega t) \\
r\omega\cos(\omega t) \\
0
\end{bmatrix}.
\]

This velocity signal is fed directly into the Jacobian IK loop as 
$\dot{\mathbf{x}}_{\text{des}}$ at each timestep. Frame consistency is 
maintained throughout: the Jacobian $\mathbf{J}(\mathbf{q})$ is constructed 
in the robot base frame and the desired velocity is expressed in the same 
frame, ensuring the IK equation

\[
\dot{\mathbf{q}} = \mathbf{J}_{\text{lin}}^{+}\,\dot{\mathbf{x}}_{\text{des}}
\]

is dimensionally and frame-consistent.

\subsection{Effect of Base Offset on Reachability}

The base offset $b_x$ directly determines the $X$-axis range of the seam 
in the robot base frame:

\[
x(t) \in \left[-r - b_x,\; r - b_x\right].
\]

With $b_x = 0.8~\text{m}$, this gives $x \in [-1.3,\; -0.3]~\text{m}$. 
The near side of the seam is only $0.3~\text{m}$ from the robot base, 
while the far side requires $1.3~\text{m}$ of reach. This asymmetry has 
a direct consequence for the RRP configuration, whose prismatic extension 
joint is bounded to $[0.1,\; 0.8]~\text{m}$: the joint saturates before 
the far side of the seam is reached, causing the tracking failure observed 
at $t \approx 2~\text{s}$.

\subsection{Tool Orientation}

A fixed tool orientation is specified as

\[
\mathbf{R}_{\text{tool}} = \mathbf{R}_z(0)\,\mathbf{R}_y(\pi)\,\mathbf{R}_x(0),
\]

corresponding to the welding torch pointing straight down toward the 
workpiece surface. The desired angular velocity is set to zero throughout:

\[
\boldsymbol{\omega}_{\text{des}} = \mathbf{0}.
\]

The full desired task-space velocity vector is assembled as

\[
\mathbf{v}_{\text{des}}(t) =
\begin{bmatrix}
\boldsymbol{\omega}_{\text{des}} \\
\dot{\mathbf{p}}_{\text{robot}}(t)
\end{bmatrix}
\in \mathbb{R}^6.
\]

It should be noted that the Jacobian IK solver uses only the linear velocity 
rows of the Jacobian ($\mathbf{J}_{\text{lin}} = \mathbf{J}(4:6,:)$), so 
tool orientation is not actively controlled during trajectory tracking. The 
end-effector orientation at any given timestep is determined entirely by the 
forward kinematics at the current joint configuration $\mathbf{q}$, and is 
not guaranteed to match $\mathbf{R}_{\text{tool}}$. This is a known 
limitation of the position-only IK formulation adopted in this work.

% =========================================================
\section{Initial Configuration Search}
% =========================================================

A valid starting configuration $\mathbf{q}_0$ is found using MATLAB's
\texttt{inverseKinematics} solver, initialised from a posture hint
$\mathbf{q}_{\text{hint}}$ --- a manually specified joint vector that
places the arm in a geometrically sensible pose (base yaw toward the
seam, prismatic near $h_t$, elbow bent). The hint does not need to be
exact; it guides the gradient-based solver away from degenerate local
minima.

\subsection{Assumptions}

The search assumes prior knowledge of the workpiece geometry and its pose
relative to the robot base frame. The seam entry point $\mathbf{p}^*$ is
computed analytically from known parameters:
\[
\mathbf{p}^* = \begin{bmatrix} r - b_x \\ 0 \\ h_t \end{bmatrix},
\]
where $r = 0.5~\text{m}$, $h_t = 1.0~\text{m}$, and $b_x = 1.2~\text{m}$.
The entry angle is fixed at $\phi_0 = 0$. In a physical deployment this
information would come from a calibrated fixture or perception system.
End-effector orientation at $\mathbf{q}_0$ is not enforced --- the IK
loop recovers correct tracking from the first timestep onward.

\subsection{MATLAB IK Solver}

The target is passed as a $4\times4$ homogeneous transform with
position-only task weights $\mathbf{w} = [0,0,0,1,1,1]$, consistent
with the position-only IK loop that follows. Once a solution is returned,
the residual error $e_0 = \|\mathbf{p}_{ee}(\mathbf{q}_0) -
\mathbf{p}^*\|_2$ and manipulability $w_0$ are verified. Warnings are
raised if $e_0 > 0.05~\text{m}$ or $w_0 < 0.05$.


% =========================================================
\section{Geometric Jacobian Construction}
% =========================================================

The geometric Jacobian $\mathbf{J}(\mathbf{q}) \in \mathbb{R}^{6 \times m}$ 
relates the active joint velocities $\dot{\mathbf{q}} \in \mathbb{R}^m$ to 
the end-effector spatial velocity:
\[
\begin{bmatrix} \boldsymbol{\omega}_{ee} \\ \dot{\mathbf{p}}_{ee} 
\end{bmatrix} = \mathbf{J}(\mathbf{q})\, \dot{\mathbf{q}},
\]
where $\boldsymbol{\omega}_{ee}$ is the angular velocity and 
$\dot{\mathbf{p}}_{ee}$ is the linear velocity of the end-effector, both 
expressed in the base frame. The Jacobian is computed from first principles 
via direct column construction --- no built-in Jacobian functions are used.

\subsection{Forward Kinematics Pass}

The first step is to evaluate the full forward kinematics chain at the 
current joint vector $\mathbf{q}$. For each DH row $i = 1,\ldots,n$, the 
cumulative base-to-frame transform is computed:
\[
{}^{0}T_i = {}^{0}T_{i-1} \cdot \texttt{spatialTransform}
\big([\alpha_{i-1},\ a_{i-1},\ d_i(\mathbf{q}),\ \theta_i(\mathbf{q})]\big),
\]
where the joint variable $q_k$ is substituted into the appropriate 
parameter depending on joint type:
\[
\theta_i = \theta_{0,i} + q_k \quad (\text{revolute}), \qquad
d_i = d_{0,i} + q_k \quad (\text{prismatic}).
\]
Fixed joints use constant DH parameters with no joint variable. All 
transforms $\{{}^{0}T_0,\ {}^{0}T_1,\ \ldots,\ {}^{0}T_n\}$ are stored, 
and the end-effector position is extracted from the final transform:
\[
\mathbf{p}_e = {}^{0}T_n(1:3,\, 4).
\]

\subsection{Column Construction}

With all frame transforms available, each active joint $k$ contributes 
one column to $\mathbf{J}$. Under Craig's modified DH convention, the 
joint axis and origin for joint $k$ are taken from ${}^{0}T_k$ --- the 
transform \textit{after} the joint's $\alpha$ twist has been applied:
\[
\hat{z}_k = {}^{0}T_k(1:3,\, 3), \qquad
\mathbf{p}_k = {}^{0}T_k(1:3,\, 4).
\]

The Jacobian column for joint $k$ is then:

\[
\mathbf{J}_k =
\begin{cases}
\begin{bmatrix} \hat{z}_k \\ \hat{z}_k \times (\mathbf{p}_e - \mathbf{p}_k) 
\end{bmatrix} & \text{revolute joint}, \\[10pt]
\begin{bmatrix} \mathbf{0} \\ \hat{z}_k \end{bmatrix} 
& \text{prismatic joint}.
\end{cases}
\]

\paragraph{Revolute joints} rotate about $\hat{z}_k$, producing both an 
angular velocity contribution $\hat{z}_k$ and a linear velocity 
contribution $\hat{z}_k \times (\mathbf{p}_e - \mathbf{p}_k)$, which is 
the moment arm cross product between the joint axis and the vector from 
the joint origin to the end-effector.

\paragraph{Prismatic joints} translate along $\hat{z}_k$, producing only 
a linear velocity contribution $\hat{z}_k$ with no angular component.

\paragraph{Fixed joints} contribute no column --- they are skipped entirely.

The full Jacobian is assembled column by column, yielding:
\[
\mathbf{J} = \begin{bmatrix} \mathbf{J}_{\omega} \\ \mathbf{J}_{v} 
\end{bmatrix} \in \mathbb{R}^{6 \times m},
\]
where the upper three rows $\mathbf{J}_{\omega}$ carry angular velocity 
contributions and the lower three rows $\mathbf{J}_{v} = \mathbf{J}(4:6,:)$ 
carry linear velocity contributions.

\subsection{Why Direct Column Construction}

The direct column construction approach was chosen over recursive velocity 
propagation because the FK transforms are already computed in the first 
pass, making the column extraction essentially free. Additionally, mixed 
revolute-prismatic chains are handled cleanly since each joint type is an 
independent column formula with no interaction between joints. Both 
approaches produce identical Jacobians.

\subsection{Singular Configurations}

A configuration $\mathbf{q}$ is singular when the Jacobian loses rank --- 
that is, when one or more joints become linearly dependent and the 
end-effector loses the ability to move in certain directions. This is 
detected using the manipulability index:
\[
w(\mathbf{q}) = \sqrt{\left|\det\!\left(
\mathbf{J}_v \mathbf{J}_v^{\top}\right)\right|}.
\]
When $w \approx 0$, the linear Jacobian $\mathbf{J}_v$ is rank-deficient 
and the standard pseudoinverse would produce arbitrarily large joint 
velocities. For the modified JetArm, singular configurations arise when:

\begin{itemize}
    \item The shoulder and elbow joints align such that links $l_1$ and 
    $l_2$ are fully extended or fully folded, reducing the effective 
    planar reach to a single line.
    \item The wrist joints align the wrist pitch and roll axes, causing 
    gimbal-lock-like loss of an orientation degree of freedom (wrist 
    singularity).
    \item The prismatic lift drives the arm to a configuration where 
    multiple revolute axes become coplanar.
\end{itemize}

In practice, the manipulability index remained between $0.8$ and $1.8$ 
throughout the welding trajectory (Figure~\ref{fig:singularity}), well 
above the detection threshold $\epsilon = 0.01$, confirming that the 
chosen base placement and joint configuration avoided singular poses 
for the full $2\pi$ seam.

% =========================================================
\section{Main Simulation Loop (Jacobian IK Integration)}
% =========================================================

The core simulation implements a discrete-time inverse kinematics loop 
in which the desired end-effector velocity is mapped to joint velocities 
at each timestep via the Jacobian pseudoinverse, and joint positions are 
updated by numerical integration. The loop runs for $N$ timesteps with 
a fixed timestep $\Delta t$, both defined by the welding path. At each 
step the Jacobian is recomputed at the current joint configuration, so 
the velocity mapping stays accurate as the robot moves.

\subsection{Logged Signals and Preallocation}

Let $m$ be the number of active (non-fixed) joints. The following arrays 
are preallocated before the loop begins:

\begin{itemize}
    \item $\mathbf{q}_{\text{hist}} \in \mathbb{R}^{m \times N}$ — joint positions at each timestep
    \item $\dot{\mathbf{q}}_{\text{hist}} \in \mathbb{R}^{m \times N}$ — joint velocities at each timestep
    \item $\mathbf{p}_{\text{ee,hist}} \in \mathbb{R}^{3 \times N}$ — actual end-effector position, used to measure tracking error
    \item $\lambda_{\text{hist}} \in \mathbb{R}^{1 \times N}$ — damping coefficient, non-zero only near singularities
    \item $w_{\text{hist}} \in \mathbb{R}^{1 \times N}$ — manipulability index at each timestep
    \item $\texttt{singular\_hist} \in \{0,1\}^{1 \times N}$ — flag indicating whether a singularity was detected
\end{itemize}

\subsection{Per-Timestep Execution}

The following steps are executed for each timestep $i = 1,\ldots,N$:

\begin{enumerate}

    \item \textbf{Store current joint state.}
    $\mathbf{q}$ is recorded at the start of the timestep before any 
    update is applied.

    \item \textbf{Forward kinematics.}
    The current end-effector position is obtained by evaluating forward 
    kinematics at the current $\mathbf{q}$:
    \[
    \mathbf{p}_{ee}(i) = {}^{0}T_{ee}(\mathbf{q})\,(1:3,\,4).
    \]
    This gives the \textit{actual} position of the end-effector, which 
    is compared against the desired position on the seam to measure drift.

    \item \textbf{Geometric Jacobian computation.}
    The geometric Jacobian $\mathbf{J}(\mathbf{q}) \in \mathbb{R}^{6 
    \times m}$ is computed from first principles at the current 
    $\mathbf{q}$:
    \[
    \mathbf{J} = \begin{bmatrix}\mathbf{J}_{\omega} \\ 
    \mathbf{J}_{v}\end{bmatrix},
    \]
    where $\mathbf{J}_{\omega}$ maps joint velocities to end-effector 
    angular velocity and $\mathbf{J}_{v} = \mathbf{J}(4:6,:)$ maps joint 
    velocities to end-effector linear velocity. Since this project tracks 
    position only, only $\mathbf{J}_{v}$ is used in the IK solve.

    \item \textbf{Velocity command with drift correction.}
    
    If joint velocities are integrated over many timesteps, small errors 
    accumulate and the end-effector gradually drifts off the seam. To 
    prevent this, a proportional correction term is added to the desired 
    velocity at each step:
    \[
    \mathbf{v}_{\text{cmd}}(i) = 
    \underbrace{\dot{\mathbf{x}}_{\text{des}}(i)}_{\text{follow the path}} 
    + \underbrace{k_p\!\left(\mathbf{p}_{\text{des}}(i) - 
    \mathbf{p}_{ee}(i)\right)}_{\text{correct the drift}},
    \quad k_p = 2.0.
    \]
    The first term drives the robot along the seam at the desired velocity. 
    The second term pulls the end-effector back toward the desired position 
    whenever it has drifted away. Without this correction, open-loop 
    integration would cause the tracking error to grow unboundedly over 
    the full $2\pi$ trajectory.

    \item \textbf{DLS pseudoinverse.}
    Joint velocities are resolved from $\mathbf{v}_{\text{cmd}}$ via the 
    Damped Least Squares (DLS) pseudoinverse:
    \[
    \dot{\mathbf{q}} =
    \mathbf{J}_{v}^{\top}
    \left(
    \mathbf{J}_{v}\mathbf{J}_{v}^{\top}
    + \lambda^2 \mathbf{I}
    \right)^{-1}
    \mathbf{v}_{\text{cmd}}.
    \]
    Near singular configurations, $\det(\mathbf{J}_v \mathbf{J}_v^\top) 
    \approx 0$ and the standard pseudoinverse produces very large joint 
    velocities. The damping term $\lambda^2\mathbf{I}$ prevents this by 
    regularising the inversion. The damping coefficient is adaptive:
    \[
    \lambda =
    \begin{cases}
    \lambda_{\max}\!\left(1 - \left(\dfrac{w}{\epsilon}\right)^{\!2}
    \right), & w < \epsilon, \\
    0, & w \ge \epsilon,
    \end{cases}
    \]
    where
    \[
    w = \sqrt{\left|\det\!\left(\mathbf{J}_{v}\mathbf{J}_{v}^{\top}
    \right)\right|}
    \]
    is the manipulability index. When $w$ is large the robot is far from 
    singularity and $\lambda = 0$, so the standard pseudoinverse is used. 
    As $w$ drops toward zero, $\lambda$ increases and the inversion is 
    progressively damped, trading a small amount of tracking accuracy for 
    bounded joint velocities.

    \item \textbf{Clamping and integration.}
    Joint velocities are clamped to their hardware limits before 
    integration:
    \[
    \dot{\mathbf{q}} \leftarrow
    \mathrm{clamp}(\dot{\mathbf{q}},\, -\mathbf{v}_{\max},\, 
    \mathbf{v}_{\max}).
    \]
    The joint state is then updated by forward Euler and clamped to 
    position limits:
    \[
    \mathbf{q}_{k+1} =
    \mathrm{clamp}\!\left(
    \mathbf{q}_k + \dot{\mathbf{q}}_k\,\Delta t,\;
    \mathbf{q}_{\min},\; \mathbf{q}_{\max}
    \right).
    \]
    Forward Euler is used because the timestep $\Delta t = 0.05~\text{s}$ 
    is small relative to the path dynamics, and the drift correction term 
    actively compensates for any integration error that accumulates.

    \item \textbf{Log outputs.}
    $\dot{\mathbf{q}}$, $\lambda$, $w$, and the singularity flag are 
    stored for post-processing and visualisation.

\end{enumerate}


\subsection{Joint Velocity Limits}

Joint velocity limits $\mathbf{v}_{\max}$ are defined in \texttt{buildConfig}
for each configuration. They were derived empirically: the simulation was
first run unconstrained, peak joint velocities were recorded, and limits
were set to $1.5\times$ the observed maximum to allow margin for transient
demands while preventing unrealistic joint speeds. For the JetArm configuration, velocity limits were set conservatively 
high at $10~\text{rad/s}$ (or $\text{m/s}$ for the prismatic joint), 
as the observed peak velocities during the welding task were well below 
$0.5~\text{rad/s}$, making the limits non-binding throughout the trajectory.
\clearpage
\section{Results and Discussion}

\subsection{End-Effector Tracking Performance}


\begin{figure}[!ht]
    \centering
    \begin{minipage}{0.55\textwidth}
        \centering
        \includegraphics[width=\linewidth]{eef_error_tracking.png}
        \caption{End-effector tracking performance: 3D path, per-axis 
        position traces, and total position error.}
        \label{fig:ee_tracking}
    \end{minipage}
    \hfill
    \begin{minipage}{0.40\textwidth}
        \centering
        \includegraphics[width=\linewidth]{robot_result.png}
        \caption{JetArm welding simulation: robot pose at mid-trajectory 
        with traced path (red) and desired seam (black dashed).}
        \label{fig:simulation}
    \end{minipage}
\end{figure}
\FloatBarrier

The modified JetArm (PRRRRR) configuration successfully traced the full 
$2\pi$ welding seam. Figure~\ref{fig:ee_tracking} shows the end-effector 
path against the desired seam in 3D, along with per-axis position traces 
and the total position error over time.



The tracking error starts at approximately $75~\text{mm}$ at $t = 0$, 
which reflects the gap between the initial configuration $\mathbf{q}_0$ 
found by the search and the exact seam entry point. The proportional 
drift correction term ($k_p = 2.0$) drives this error down rapidly — 
within approximately $2~\text{s}$ the error falls below $5~\text{mm}$ 
and remains there for the rest of the trajectory. The maximum Z deviation 
over the full run is:
\[
\max_t \left| z_{\text{ee}}(t) - h_t \right| = 16.5~\text{mm},
\]
which occurs during the initial transient. Once the feedback correction 
has converged, the Z deviation is negligible. The X and Y traces overlap 
the desired seam closely for the entire trajectory, confirming that the 
circular path is traced accurately in the horizontal plane.


\clearpage
\subsection{Joint Trajectories}

\begin{figure}[!ht]
    \centering
    \begin{minipage}{0.49\textwidth}
        \centering
        \includegraphics[width=\linewidth]{jointangles.png}
        \caption{Joint positions vs time.}
        \label{fig:joint_angles}
    \end{minipage}
    \hfill
    \begin{minipage}{0.49\textwidth}
        \centering
        \includegraphics[width=\linewidth]{jointvelocities.png}
        \caption{Joint velocities vs time.}
        \label{fig:joint_velocities}
    \end{minipage}
\end{figure}
\FloatBarrier

Joints $q_1$ through $q_5$ exhibit smooth, continuous position and 
velocity profiles consistent with the sinusoidal velocity demand of the 
circular path. Key observations from the plots:

The prismatic lift joint $q_1$ oscillates between approximately 
$0.80~\text{m}$ and $1.10~\text{m}$, well within its limits of 
$[-0.5,\, 1.5]~\text{m}$, confirming that vertical reach is not a 
bottleneck for this task. The base yaw $q_2$ sweeps between approximately 
$2.6$ and $3.6~\text{rad}$, and the shoulder $q_3$ and elbow $q_4$ vary 
smoothly over ranges of roughly $1.5~\text{rad}$ each. All velocities 
remain well below the set limits of $10~\text{rad/s}$, with peak values 
under $0.4~\text{rad/s}$ for the revolute joints and under $0.1~\text{m/s}$ 
for the prismatic joint.

The wrist roll joint $q_6$ is an exception. Its velocity profile shows 
high-frequency noise at the order of $10^{-17}~\text{rad/s}$, and its 
position remains effectively constant throughout the trajectory. This 
occurs because the robot has 6 active joints controlling only 3 
task-space coordinates --- the system is kinematically redundant with 
respect to the task. The wrist roll axis lies in the null space of 
$\mathbf{J}_v$ for much of the trajectory, meaning it has no influence 
on end-effector position. The DLS pseudoinverse distributes numerical 
residuals into this unconstrained direction, producing floating-point 
chatter at machine epsilon. This has no physical significance and no 
effect on tracking performance.
\clearpage
\subsection{Singularity Analysis}

\begin{figure}[!ht]
    \centering
    \includegraphics[width=0.75\linewidth]{singularity.png}
    \caption{Manipulability index $w$ and DLS damping coefficient 
    $\lambda$ over time.}
    \label{fig:singularity}
\end{figure}
\FloatBarrier

The manipulability index $w$ remained well above the singularity 
threshold $\epsilon = 0.01$ throughout the entire trajectory, ranging 
from approximately $0.8$ at the start to a peak of $1.8$ at mid-trajectory, 
then returning to approximately $0.8$ at the end. The DLS damping 
coefficient $\lambda$ was zero for all 252 timesteps, confirming that no 
singular or near-singular configurations were encountered:
\[
\text{Singularity events: } 0 \text{ / } 252 \text{ timesteps.}
\]
The variation in $w$ over the trajectory reflects the changing arm 
geometry as it sweeps around the cylinder --- the arm is better conditioned 
when extended outward (mid-trajectory) than when pointing toward the near 
or far sides of the seam. Importantly, $w$ never drops close to the 
threshold, which means the DLS damping mechanism was never needed. This 
indicates that the chosen base offset $b_x = 1.2~\text{m}$ and joint 
configuration kept the arm in well-conditioned poses throughout.

\subsection{Summary}

The modified JetArm successfully completed the full welding task with:
\begin{itemize}
    \item Full $2\pi$ seam traced with no interruption
    \item Maximum position error of $75~\text{mm}$ at $t=0$, converging 
    to under $5~\text{mm}$ within $2~\text{s}$
    \item Maximum Z deviation of $16.5~\text{mm}$ during initial transient
    \item Zero singularity events across all 252 timesteps
    \item No joint limit violations
    \item Smooth joint trajectories with no abrupt velocity changes
\end{itemize}

The initial transient error is a direct consequence of the initial 
configuration search not enforcing exact position at $\mathbf{q}_0$ --- 
the feedback correction term handles the convergence instead. This is an 
acceptable trade-off: requiring exact initialisation would demand a 
config-specific analytical IK, while the current approach remains 
configuration-agnostic and converges reliably within the first few seconds.

\section{Challenges and Limitations}

\subsection{Challenges Encountered}


\paragraph{End-Effector Drift}
The original IK loop was purely velocity-tracking with no position feedback. 
Small integration errors from the forward Euler scheme accumulated over 
time, causing the end-effector to drift progressively off the seam. This 
is visible in Figure~\ref{fig:challenge1}, where the Z axis deviates 
significantly from $h_t = 1.0~\text{m}$ and the total error grows to 
over $700~\text{mm}$ mid-trajectory. The fix was to add a proportional 
feedback correction term $k_p(\mathbf{p}_{\text{des}} - \mathbf{p}_{ee})$ 
to the velocity command at each timestep, which actively pulled the 
end-effector back toward the seam and reduced steady-state error to 
under $5~\text{mm}$.

\begin{figure}[!ht]
    \centering
    \includegraphics[width=0.95\linewidth]{challenge_2.png}
    \caption{Failed tracking run before drift correction was added. 
    The Z axis deviates significantly from $h_t = 1.0~\text{m}$ and 
    the total position error exceeds $700~\text{mm}$ mid-trajectory.}
    \label{fig:challenge1}
\end{figure}
\FloatBarrier



\subsection{Limitations of the Approach}

\paragraph{Position-Only IK}
The IK solver uses only the linear velocity rows of the Jacobian 
($\mathbf{J}_v$), so end-effector orientation is never actively 
controlled. The welding tool orientation at any given timestep is 
whatever forward kinematics happens to give at the current joint 
configuration, and is not guaranteed to match the desired tool-down 
orientation $\mathbf{R}_{\text{tool}}$. For a real welding application, 
incorrect tool orientation would produce a defective weld. A full 6-DOF 
task-space controller using the complete Jacobian would be required to 
enforce both position and orientation simultaneously.

\paragraph{Forward Euler Integration}
Joint positions are updated using forward Euler, which is the simplest 
but least accurate integration scheme. It introduces truncation error at 
each timestep, which the proportional feedback term compensates for but 
does not eliminate. A higher-order integration scheme such as Runge-Kutta 
would reduce integration error without requiring feedback correction.

\paragraph{Kinematic Redundancy Not Exploited}
The modified JetArm has 6 active joints controlling 3 task-space 
coordinates, leaving a 3-dimensional null space. The DLS pseudoinverse 
resolves this by finding the minimum-norm joint velocity solution, which 
minimises joint motion but makes no attempt to use the redundancy 
productively --- for example, to avoid self-collision, optimise 
manipulability, or keep joints away from their limits. In the current 
implementation the wrist roll joint ($q_6$) is effectively frozen 
throughout the trajectory at whatever value the initial configuration 
happened to give it. Null-space control techniques such as the 
gradient projection method could be used to exploit this redundancy.

\paragraph{No Self-Collision Detection}
The simulation has no collision checking. The arm's minimum-norm posture 
resolution occasionally results in configurations where links are 
geometrically close to or intersecting each other. In a physical 
deployment, a collision-aware motion planner would be required.

\section{Conclusion}

A modular, configuration-agnostic Jacobian IK framework was implemented 
for robotic welding path tracking. The system builds robot models 
dynamically from Modified DH parameters, computes the geometric Jacobian 
from first principles at each timestep, and resolves joint velocities via 
DLS pseudoinverse with adaptive singularity damping and proportional drift 
correction.

The modified JetArm (PRRRRR) configuration demonstrated reliable full-seam 
tracking with sub-5mm steady-state error and zero singularity events, 
validating the control architecture. The initial tracking transient 
($\sim 75~\text{mm}$) highlights the dependency on initialisation quality, 
and motivates future work on tighter initial configuration constraints or 
a warm-start analytical solution for the first timestep.



\end{document}